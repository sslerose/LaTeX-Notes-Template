\newcounter{chapter}
\newcounter{mysection}[chapter]

\newcommand{\chapter}[1]{%
  \clearpage
  \stepcounter{chapter}

  \begingroup           % Begin local group to contain font size/style changes
  
  \setlength{\leftskip}{0pt}
  \noindent\bfseries\Huge\thechapter\par\vspace{-12pt}
  \sbox0{\LARGE{#1}}    % Store the text box to calc. its width
  \noindent\rule[\dimexpr\ht\strutbox-26pt]{\wd0}{0.4pt}\par    % First rule with width of sbox0
  \noindent\usebox0\par\vspace{6pt}     % Display text from sbox0
  \noindent\rule[\dimexpr\ht\strutbox]{\wd0}{0.4pt}             % Second rule using same width
  
  \endgroup             % End local group
}


% Second input used to declare a non-step increase in section numbering (i.e., section 1 declared with \section{Title} followed by section 3 declared with \section{Title}{3}, calling \section again resumes normal step numbering with section 4)
\renewcommand{\section}[2]{
  \ifx\relax#2\relax
      \refstepcounter{mysection}
    \else
      \setcounter{mysection}{#2}
    \fi

  \setcounter{equation}{0}
  \begingroup % Begin local group to contain font size/style changes
  \large
  \setlength{\leftskip}{0pt}

  \noindent\textbf{\thechapter-\themysection} % Number aligned with default left margin
  \hspace{-2em} % Indent section title 2em from the default margin
  \hspace{+1.6em} % Adjust back to the left margin
  \textbf{#1} % Section title
  
  \endgroup % End local group
  \medskip
}

% Declare equation numbering as (Ch.Sec.Num)
\renewcommand{\theequation}{\thechapter.\themysection.\arabic{equation}}